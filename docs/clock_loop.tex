\documentclass{article}

\usepackage{amsmath}
\usepackage{tikz}
\usetikzlibrary{dsp,chains}

\DeclareMathAlphabet{\mathpzc}{OT1}{pzc}{m}{it}
\newcommand{\z}{\mathpzc{z}}

\title{Symbol Clock Phase Tracking Loop}
\date{2017-04-08}
\author{Andy Walls}

\begin{document}

\pagenumbering{gobble}
\maketitle
\newpage

\pagenumbering{arabic}

\begin{figure}
  \begin{tikzpicture}
    \matrix[row sep=2.5mm, column sep=10mm]
    {
      \node[dspnodeopen, dsp/label=above] (m00) {$\theta_i[n]$}; &
      \node[dspmultiplier, dsp/label=above]    (m01) {PD};            &
      \node[dspnodefull, dsp/label=above] (m02) {$e[n]$};        &
      \node[dspsquare]                    (m03) {$\alpha$};      &
      \node[coordinate]                   (m04) {};              &
      \node[coordinate]                   (m05) {};              &
      \node[coordinate]                   (m06) {};              &
      \node[coordinate]                   (m07) {};              \\

      \node[coordinate]                   (m10) {};              &
      \node[coordinate]                   (m11) {};              &
      \node[coordinate]                   (m12) {};              &
      \node[coordinate]                   (m13) {};              &
      \node[coordinate]                   (m14) {};              &
      \node[coordinate]                   (m15) {};              &
      \node[coordinate]                   (m16) {};              &
      \node[coordinate]                   (m17) {};              \\

      \node[coordinate]                   (m20) {};              &
      \node[coordinate]                   (m21) {};              &
      \node[coordinate]                   (m22) {};              &
      \node[dspsquare]                    (m23) {$\beta$};       &
      \node[dspadder]                     (m24) {};              &
      \node[coordinate]                   (m25) {};              &
      \node[dspnodefull, dsp/label=above] (m26) {$\bar{\theta}_{inc}[n]$};  &
      \node[dspadder]                     (m27) {};              \\

      \node[coordinate]                   (m30) {};              &
      \node[coordinate]                   (m31) {};              &
      \node[coordinate]                   (m32) {};              &
      \node[coordinate]                   (m33) {};              &
      \node[coordinate]                   (m34) {};              &
      \node[dspsquare]                    (m35) {$\z^{-1}$};     &
      \node[coordinate]                   (m36) {};              &
      \node[coordinate]                   (m37) {};              \\

      \node[coordinate]                   (m40) {};              &
      \node[coordinate]                   (m41) {};              &
      \node[coordinate]                   (m42) {};              &
      \node[coordinate]                   (m43) {};              &
      \node[coordinate]                   (m44) {};              &
      \node[coordinate]                   (m45) {};              &
      \node[coordinate]                   (m46) {};              &
      \node[coordinate]                   (m47) {};              \\

      \node[coordinate]                   (m50) {};              &
      \node[coordinate]                   (m51) {};              &
      \node[dspnodefull, dsp/label=above] (m52) {$\theta_o[n]$}; &
      \node[dspsquare]                    (m53) {$\z^{-1}$};     &
      \node[coordinate]                   (m54) {};              &
      \node[dspadder]                     (m55) {};              &
      \node[coordinate]                   (m56) {};              &
      \node[coordinate]                   (m57) {};              \\

      \node[coordinate]                   (m60) {};              &
      \node[coordinate]                   (m61) {};              &
      \node[coordinate]                   (m62) {};              &
      \node[coordinate]                   (m63) {};              &
      \node[coordinate]                   (m64) {};              &
      \node[coordinate]                   (m65) {};              &
      \node[coordinate]                   (m66) {};              &
      \node[coordinate]                   (m67) {};              \\
    };
    \begin{scope}[start chain]
      \chainin (m00);
      \chainin (m01) [join=by dspconn];
      \chainin (m03) [join=by dspconn];
      \chainin (m07) [join=by dspline];
      \chainin (m27) [join=by dspconn];
      \chainin (m57) [join=by dspline];
      \chainin (m56) [join=by dspline, label=$\hat{\theta}_{inc}{[n]}$ ];
      \chainin (m55) [join=by dspconn];
      \chainin (m54) [join=by dspline, label=$\theta_o{[n+1]}$ ];
      \chainin (m53) [join=by dspconn];
      \chainin (m51) [join=by dspline];
      \chainin (m11) [join=by dspline];
      \chainin (m01) [join=by dspconn];
    \end{scope}
    \begin{scope}[start chain]
      \chainin (m02);
      \chainin (m22) [join=by dspline];
      \chainin (m23) [join=by dspconn];
      \chainin (m24) [join=by dspconn];
      \chainin (m27) [join=by dspconn];
    \end{scope}
    \begin{scope}[start chain]
      \chainin (m26);
      \chainin (m36) [join=by dspline];
      \chainin (m35) [join=by dspconn];
      \chainin (m34) [join=by dspline];
      \chainin (m24) [join=by dspconn];
    \end{scope}
    \begin{scope}[start chain]
      \chainin (m52);
      \chainin (m62) [join=by dspline];
      \chainin (m65) [join=by dspline];
      \chainin (m55) [join=by dspconn];
    \end{scope}
  \end{tikzpicture}
  \caption{Symbol Clock Phase Tracking Loop Model}
\end{figure}

The symbol clock tracking loop, with an ideal clock phase error detector and a
PI filter, has a discrete time difference equation in terms of the phase error,
$e[n]$, proportional gain, $\alpha$, and the integral gain, $\beta$, as follows:

\begin{align*}
   e[n] &= \theta_i[n] - \theta_o[n] \\
   \theta_o[n+1] &= \theta_o[n] + \alpha e[n] + \sum_{k=0}^{n}{\beta e[n-k]}\\
   \theta_o[n+1] &= \theta_o[n]
                  + \alpha\theta_i[n] - \alpha\theta_o[n]
                  + \beta\sum_{k=0}^{n}{\theta_i[n-k]}
                  - \beta\sum_{k=0}^{n}{\theta_o[n-k]}\\
\end{align*}

Assuming $n \rightarrow \infty$ and taking the Z transform:

\begin{align*}
    z&\Theta_o(z) = \Theta_o(z) + \alpha\Theta_i(z) - \alpha\Theta_o(z)
               + \beta\frac{z}{z-1}\Theta_i(z) - \beta\frac{z}{z-1}\Theta_o(z)\\
    \\
    &\Theta_o(z)\left[z - 1 + \alpha + \beta\frac{z}{z-1}\right]
     = \Theta_i(z)\left[\alpha + \beta\frac{z}{z-1}\right]\\
\end{align*}

After some algebraic manipulation, one arrives at the following symbol clock
tracking loop phase-transfer function, in terms of the proportional gain,
$\alpha$, and the integral gain, $\beta$:

\begin{align*}
   H(z) &= \dfrac {\Theta_o(z)}{\Theta_i(z)}
         = (\alpha + \beta)z^{-1} \cdot
          \dfrac{
                 1
                 - \dfrac{\alpha}{\alpha + \beta} z^{-1}
                }
                {
                 1
                 - 2 \left(1 - \dfrac{\alpha + \beta}{2}\right) z^{-1}
                 + (1 - \alpha) z^{-2}
                } \\
\end{align*}

With Z-plane zeros and poles:

\begin{align*}
    z_{1} &= \frac{\alpha}{\alpha+\beta} \\
    z_{2} &\rightarrow \infty \\
    p_{1,2} &= \left(1-\frac{\alpha+\beta}{2}\right) \pm
           \sqrt{\left(1-\frac{\alpha+\beta}{2}\right)^{2}
                 - \left(1-\alpha\right)}\\
\end{align*}

Mapping the above phase-transfer function to the standard form of a transfer
function for an analog second order control loop mapped to the digital domain
with the mapping $z = e^{sT}$ applied to the s-plane poles,
$s_{1,2} = -\zeta\omega_{n} \pm \omega_{n}\sqrt{\zeta^{2}-1}$, one obtains an
alternate form of the transfer function, directly related to the damping factor 
$\zeta$, the natural radian frequency $\omega_{n}$, the damped radian frequency
of oscillation $\omega_{d}$, and the symbol clock period $T$:

\begin{align*}
   H(z) &=
      \begin{cases}
         \dfrac{
                [2 -2\cos(\omega_{d}T)e^{-\zeta\omega_{n}T}] z 
                -2\sinh(\zeta\omega_{n}T)e^{-\zeta\omega_{n}T} 
               }
               {
                z^{2}
                - 2 \cos(\omega_{d}T) e^{-\zeta\omega_{n}T} z
                + e^{-2\zeta\omega_{n}T}
               }
               & \quad \text{for} \quad \zeta < 1 \quad \text{with}
               \quad \omega_{d}T = \omega_{n}T \sqrt{1 - \zeta^{2}}
               \\
\\
         \dfrac{
                [2 -2(1)e^{-\zeta\omega_{n}T}] z 
                -2\sinh(\zeta\omega_{n}T)e^{-\zeta\omega_{n}T} 
               }
               {
                z^{2}
                - 2(1)e^{-\zeta\omega_{n}T} z
                + e^{-2\zeta\omega_{n}T}
               }
               & \quad \text{for} \quad \zeta = 1 \quad \text{with}
               \quad \omega_{d}T = 0
               \\
\\
         \dfrac{
                [2 -2\cosh(\omega_{d}T)e^{-\zeta\omega_{n}T}] z 
                -2\sinh(\zeta\omega_{n}T)e^{-\zeta\omega_{n}T} 
               }
               {
                z^{2}
                - 2 \cosh(\omega_{d}T) e^{-\zeta\omega_{n}T} z
                + e^{-2\zeta\omega_{n}T}
               }
               & \quad \text{for} \quad \zeta > 1 \quad \text{with}
               \quad \omega_{d}T = \omega_{n}T \sqrt{\zeta^{2} - 1}
               \\
      \end{cases}
\\
\end{align*}

The PI filter gains, expressed in terms of the damping factor $\zeta$,
the natural radian frequency $\omega_{n}$, the damped radian frequency of
oscillation $\omega_{d}$, and the symbol clock period $T$ are:

\begin{align*}
   \alpha &= 2e^{-\zeta\omega_{n}T} \sinh(\zeta\omega_{n}T) \\
\\
   \beta  &=
      \begin{cases}
         2
         -2e^{-\zeta\omega_{n}T} [\sinh(\zeta\omega_{n}T) + \cos(\omega_{d}T)] &
         \text{for} \quad \zeta < 1 \quad (under \: damped)\\
         2
         -2e^{-\zeta\omega_{n}T} [\sinh(\zeta\omega_{n}T) + 1] &
         \text{for} \quad \zeta = 1 \quad (critically \: damped)\\
         2
         -2e^{-\zeta\omega_{n}T} [\sinh(\zeta\omega_{n}T) +\cosh(\omega_{d}T)] &
         \text{for} \quad \zeta > 1 \quad (over \: damped)\\
      \end{cases} \\
\\
\end{align*}

The digital loop bandwith of an underdamped symbol clock tracking loop is
approximately $\omega_{n}$.  However the symbol clock period, $T$, is being
estimated by the symbol clock tracking loop and can vary over time, so that
setting the loop bandwidth directly can be a problem.  To avoid that difficulty,
one can specify the normalized loop bandwidth in terms of the normalized digital
natural radian frequency, $\omega_{n\_norm}$.

\begin{align*}
    \omega_{n}T = \omega_{n\_norm} = 2 \pi f_{n\_norm} = 2 \pi f_{n} T =
    \pi \dfrac{f_{n}}{\left(\dfrac{F_{c}}{2}\right)}
\end{align*}

* A note on symbol clock phase vs. interpolating resampler sample phase,
since most GNURadio symbol synchronization blocks seem to have the same
implementation error:

In general, the symbol clock phase, that the symbol clock tracking loop
estimates and tracks, cannot be used alone to derive the interpolating resampler
sample phase used in symbol synchronization, except in the very special case of
the symbol clock period being exactly divisible by the input sample stream
sample period.  Since this is never guaranteed in tracking real symbol clocks,
one should not use the symbol clock phase alone to compute the interpolating
resampler sample phase.

Consider, in the analog time domain, the optimum symbol sampling instants
$t_{k}$, of an analog input signal $x(t)$, at an optimal symbol clock
phase $\tau_{0}$ and the symbol clock period $T_{c}$:

\begin{align*}
   t_{k} &= \tau_{0} + k T_{c} \\
   y_{k} &= x(t_{k}) = x(\tau_{0} + k T_{c}) \\
\end{align*}

If one divides the $t_{k}$ times by the input sample stream sample period
$T_{i}$, the correct interpolating resampler sample phase $\tau_{0\_i}$ will
get a contribution from the term $T_{c\_remainder}$ (which is an error) as shown
below:

\begin{align*}
   \dfrac{t_{k}}{T_{i}} &= \dfrac{\tau_{0}}{T_{i}} + \dfrac{k T_{c}}{T_{i}} \\
    &= (m + \tau_{0\_remainder}) + (n + T_{c\_remainder}) \\
    &= \tau_{0\_remainder} + T_{c\_remainder} + (m + n) \\
    &= \tau_{0\_i} + k^{\prime}
\end{align*}

So instead of using the symbol clock sample phase alone to obtain the
interpolating resampler sample phase, one should use the previous interpolating
resampler sample phase and the instantaneous clock period estimate provided by
the symbol clock tracking loop.
\end{document}
